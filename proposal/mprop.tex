\documentclass{mprop}
\usepackage{graphicx}

% alternative font if you prefer
%\usepackage{times}

% for alternative page numbering use the following package
% and see documentation for commands
%\usepackage{fancyheadings}


% other potentially useful packages
%\uspackage{amssymb,amsmath}
%\usepackage{url}
%\usepackage{fancyvrb}
%\usepackage[final]{pdfpages}

\begin{document}

%%%%%%%%%%%%%%%%%%%%%%%%%%%%%%%%%%%%%%%%%%%%%%%%%%%%%%%%%%%%%%%%%%%
\title{Extending QUIC to Support RTP}
\author{Vivian Band}
\date{\today}
\maketitle
%%%%%%%%%%%%%%%%%%%%%%%%%%%%%%%%%%%%%%%%%%%%%%%%%%%%%%%%%%%%%%%%%%%

%%%%%%%%%%%%%%%%%%%%%%%%%%%%%%%%%%%%%%%%%%%%%%%%%%%%%%%%%%%%%%%%%%%
\tableofcontents
\newpage
%%%%%%%%%%%%%%%%%%%%%%%%%%%%%%%%%%%%%%%%%%%%%%%%%%%%%%%%%%%%%%%%%%%

%%%%%%%%%%%%%%%%%%%%%%%%%%%%%%%%%%%%%%%%%%%%%%%%%%%%%%%%%%%%%%%%%%%
\section{Introduction}\label{intro}

%briefly explain the context of the project problem

QUIC is a transport protocol currently being developed by Google and the IETF as an alterative to TCP, focusing on userspace development rather than kernel modifications to allow widespread deployment and for ease of changing protocol behaviours. QUIC is reported to serve up to 9.1\% of Internet traffic as of 2018, with 98\% of this being generated by Google services \cite{Ruth2018}. 

A key feature of QUIC traffic is that all content, including the majority of header fields, is encrypted by default using TLS 1.3. While there are questions over intentionally introduced vulnerabilities in this encryption scheme, QUIC's focus on encrypted traffic as standard for increased security is likely to contribute towards an increased use of the protocol, combined with backing from influential technology companies and the ease of deploying userspace protocols.

Dynamic Adaptive Streaming over HTTP (DASH) is widely used by online streaming services such as Netflix to deliver multimedia content, but this approach encounters problems with stalling due to head-of-line blocking in TCP. QUIC mitigates this issue by using several streams demultiplexed over a single UDP socket - although head-of-line blocking still occurs in response to loss, the obstruction is confined to a single stream rather than blocking the entire connection. With some adjustments to optimisation algorithms used in DASH, QUIC could be used to provide higher quality streaming services for pre-recorded content \cite{Bhat2017}.

Real-time media is commonly streamed using the Real-time Transport Protocol (RTP) over UDP. Applications using RTP prioritise low latency over reliability: the end-user experience of VoIP sessions and multiplayer games, for example, is significantly impacted by even minor delays in data reaching the application, but a small amount of packet loss may not be noticed.

Although more complicated than DASH, where media is delivered as a series of \texttt{HTTP GET} requests, RTP allows applications to customise how data should be formatted and transferred between participants through the use of RTP profiles \cite{RTP-RFC}. This flexibility allows RTP to be used for a wide range of applications, including technologies which have only started becoming widespread in recent years like AR and VR. QUIC's behaviours are similarly malleable due to not requiring any kernel modifications - while guaranteed reliability is not desirable for real-time applications, an extension to QUIC which introduces partial reliability alongside existing stream demultiplexing capabilities, framing, and increased security could be highly desirable for interactive media.

%\subsection{A subsection}
%Please note your proposal need not follow the included section headings - this is only a suggested structure. Also add subsections etc as required

%example references: \cite{BK08}

\newpage

%%%%%%%%%%%%%%%%%%%%%%%%%%%%%%%%%%%%%%%%%%%%%%%%%%%%%%%%%%%%%%%%%%%
\section{Statement of Problem}

%clearly state the problem to be addressed in your forthcoming project. Explain why it would be worthwhile to solve this problem.

Best-effort transport protocols like TCP and QUIC are not optimal choices for streaming real-time media due to prioritising reliability over timeliness. Guaranteed reliability is useful for applications with relaxed latency bounds and strict ordering requirements (eg. serving static web pages), but head-of-line blocking in response to packet loss impairs the performance of time-sensitive applications typically using RTP. This effect is noticeable even when the delay is confined to a single stream within a QUIC connection.

Creating an RTP extension for QUIC is further complicated by a semantic mismatch between the two protocols: RTP uses datagrams to transmit frame data and timing information, while QUIC is a stream-based protocol which does not have distinct boundaries between units of delivered data once payloads are removed from their frames. [Challenges in reassembling large I-frames which are not normally split between datagrams in UDP-RTP, plus challenges in identifying multiple smaller frames sent in a single QUIC packet, challenges in where to do reassembly and error correction]

The disparaties between RTP and QUIC in terms of semantics and reliability make creating an RTP extension to the protocol challenging. However, the flexibility of QUIC as a userspace protocol allows significant changes to its underlying loss recovery and stream buffering behaviours, meaning that it is possible to build a variant of QUIC suited to real-time media through use of selective retransmissions and custom payload frame types. 

[write about partial reliability and challenges in deadline awareness]

\newpage

%%%%%%%%%%%%%%%%%%%%%%%%%%%%%%%%%%%%%%%%%%%%%%%%%%%%%%%%%%%%%%%%%%%
\section{Background Survey}

present an overview of relevant previous work including articles, books, and existing software products. Critically evaluate the strengths and weaknesses of the previous work.

\newpage

%%%%%%%%%%%%%%%%%%%%%%%%%%%%%%%%%%%%%%%%%%%%%%%%%%%%%%%%%%%%%%%%%%%
\section{Proposed Approach}

state how you propose to solve the software development problem. Show that your proposed approach is feasible, but identify any risks.

\newpage

%%%%%%%%%%%%%%%%%%%%%%%%%%%%%%%%%%%%%%%%%%%%%%%%%%%%%%%%%%%%%%%%%%%
\section{Work Plan}

show how you plan to organize your work, identifying intermediate deliverables and dates.

\newpage

%%%%%%%%%%%%%%%%%%%%%%%%%%%%%%%%%%%%%%%%%%%%%%%%%%%%%%%%%%%%%%%%%%%
% it is fine to change the bibliography style if you want
\bibliographystyle{plain}
\bibliography{mprop}
\end{document}
